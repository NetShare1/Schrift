\chapter{Fazit}
\label{cha:Fazit}
\section{Error Handling im Multi-Wan Bonding fähigen Windows Treiber}
Aus dem Grund das wir großen Zeitdruck hatten, hatten wir nicht die Möglichkeit qualitativ hochwertiges Error Handling in den Multi-Wan Bonding fähigen Treiber einzubauen. Deswegen werde ich hier darüber schreiben, was ich vorhatte in den Treiber einzubauen.
\\\\
Erstens würden alle Sender und Empfänger Threads überwacht werden. Wenn einer der Threads abstürzt soll geprüft werden, ob dieser Thread abgestürzt ist, weil der Socket in diesem Thread geschlossen wurde. Wenn der Socket in diesem Thread geschlossen wurde, bedeutet das die NIC die mit diesem Socket verbunden ist deaktiviert wurde. Hier würde der Thread also nicht nochmal gestartet werden, weil diese NIC nicht mehr mit einem Netzwerk verbunden ist. Wenn jedoch ein anderer Fehler aufgetreten sein sollte, soll der Thread neu gestartet werden. Wenn ein Thread innerhalb von einer Minute mehr als dreimal wegen eines unbekannten Fehlers neu gestartet werden muss, wird ein Fehler Status für diesen Thread aufgerufen. Der Fehler Status besagt nur das etwas in diesem Thread nicht stimmt und das aus diesem Grund diese NIC nicht mehr verwendet werden kann. Das heißt, dass falls der andere Thread, der mit der NIC verbunden ist, noch läuft wird dieser gestoppt. Nachdem wird in einer Datei festgehalten das für diese NIC ein Fehlerzustand aufgerufen wurde. Wenn bei einem kompletten Neustart des Multi-Wan Bonding fähigen Treibers noch einmal ein Fehlerzustand auf der gleichen NIC festgestellt wird, wird diese NIC auf eine Blacklist gesetzt und kann nicht mehr von dem Benutzer verwendet werden.
\\\\
Der Multi-WAN Bonding fähige Treiber wird außerdem in einen Fehlerzustand versetzt, wenn keine NIC zum Verwenden ausgewählt wurde. Wenn keine NIC zum Verwenden ausgewählt wird, würde der Treiber immer noch alle IP Pakete von Windows einsammeln, könnte diese aber nicht mehr an den Multi-Wan Bonding fähigen Server schicken. Deswegen verweigert der Multi-Wan Bonding fähiger Treiber in diesem Fall das Aufbauen einer Verbindung mit dem Multi-Wan Bonding fähigen Server, und gibt einen Fehlercode zurück.