\chapter{Kurzfassung}

Diese Diplomarbeit beschäftigt sich mit der Planung und Entwicklung eines Multi-Wan Bonding Prototypen, namens NetShare. Aufgrund der Tatsache, dass Windows keine kostenfreie Möglichkeit bietet, mehrere Internetverbindungen gleichzeitig zu verwenden und deren Bandbreiten zu bündeln, haben wir uns dazu entschlossen, einen Prototyp für eine solche Softwarelösung zu entwickeln.
\\\\
Zur Umsetzung benötigen wir einen Multi-Wan Bonding fähigen Server, Treiber und eine Windows Desktop-Anwendung zur Steuerung des Treibers. Hierfür haben wir uns mit den Technologien IP Routing Table, Nating, TUN-/ TAP-Geräte, Virtual Private Networks, Winforms, und Windows Treibern auseinandergesetzt.
\\\\
Im Zuge der Arbeit erläutern wir verwendete Technologien und die Ansätze, die beim Implementieren unseres Multi-Wan Bonding Prototypen verfolgt wurden. Weiters veranschaulichen wir anhand von Beispielen, wie die einzelnen Bereiche umgesetzt wurden. 