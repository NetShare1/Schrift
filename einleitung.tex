\chapter{Einleitung}
\label{cha:Einleitung}

\section{Projekthintergrund und Idee}
In einer Zeit der rasant voranschreitenden Digitalisierung, ist eine schnelle Internetverbindung zur Voraussetzung des täglichen Alltags geworden. Aufgrund von oft mangelhafter Infrastruktur, haben wir uns dazu entschieden einen Multi-Wan Bonding Prototyp zu entwickeln, welcher die Bündelung mehrere Internetverbindungen ermöglicht. Dadurch könnte eine schnelle und stabiler Internetanbindung erreicht werden. Es soll für den Anwender mithilfe einer simplen Desktop-Anwendung möglich sein, mehrere Internetverbindungen auszuwählen und diese gebündelt nutzen zu können.
\section{Gliederung der Diplomschrift}
\paragraph{Kapitel 2 Technologien für Multi-Wan Bonding}erläutert die verwendeten Technologien die uns bei der Implementierung des Prototypen in hinsicht auf Multi-Wan Bonding geholfen haben.
\paragraph{Kapitel 3 Technologien für Windows Desktop-Anwendungen}beschreibt die Technologien die wir zum Erstellen der Windows Desktop-Anwendung verwendet haben.
\paragraph{Kapitel 4 Verwandte Arbeiten}listet andere Lösungen im Bereich Multi-Wan Bonding auf. 
\paragraph{Kapitel 5...}
\paragraph{Kapitel 6 Multi-Wan Bonding Treiber für Windows}veranschaulicht sowohl die Herangehensweise der Implementierung, als auch die Architektur des Multi-Wan Bonding Treibers.
\paragraph{Kapitel 7 Windows Desktop-Anwendung zur Steuerung des Treibers}beschreibt die Implementierung des Designs und der Funktionalität der Desktop-Anwendung, die Kommunikation mit dem Multi-Wan Bonding Treiber, sowie das Benutzen eines Windows-Services zur Crash-Recovery des Treibers.
\paragraph{Kapitel 8 Funktionstests für Multi-Wan Bonding}gibt den Testaufbau und die Testergebnisse bei unseren Integrationstests wieder.